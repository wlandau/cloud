\documentclass[10pt]{beamer} 

\usefonttheme{serif}

\usepackage{xcolor}
\usepackage{hyperref}
\usepackage{amsmath,amssymb}
\usepackage{verbatim}

\renewcommand\rmdefault{pplx}
\renewcommand\mathfamilydefault{cmr}

\title{A tutorial on AWS CodeDeploy}
\author{\underline{Xin Yin}}
\date{\scriptsize 10/14/2015\\
Cloud Computing Working Group}

\begin{document}

\frame{\titlepage}

%\section{CodeDeploy in a nutshell}
\begin{frame}
    \frametitle{CodeDeploy - Examples}
    \begin{enumerate}
        \item Alice runs a few Shiny servers on multiple EC2 instances. Now that Alice wants to deploy a new version of her Shiny application to all servers.
        \item Bob wants to run some bioinformatics tools on multiple EC2 instances to crunch his sequencing data in parallel. To do so, he needs to install some packages and software.
        \item Charlie uses multiple EC2 instances to run his simulations. He made some changes to his R code and wants to upload and rerun the analysis with the updated R program.
        \item Daniel sporadically uses spot instances for computing, and he wants to automate the process of installing R, R packages and upload his R program to newly launched instances.
    \end{enumerate}

    \vspace{0.5in}
    \pause
    {\bf NOTE}: the solutions are not {\bf unique} (AMIs, Docker, {\tt rsync}, {\tt scp}, Git-Hooks, {\tt user-data}).
\end{frame}

\begin{frame}
    \frametitle{CodeDeploy in a nutshell}
    
    \includegraphics[width=\textwidth]{architecture.png}

    {\tiny \url{http://docs.aws.amazon.com/codedeploy/latest/userguide/welcome.html}}
\end{frame}

\begin{frame}
    \frametitle{CodeDeploy - Terminology}
    \begin{itemize}
        \item Application: 

            ``{\it An application in AWS CodeDeploy is simply a unique identifier used by AWS CodeDeploy to deploy the correct revision to the correct set of instances with the correct deployment configuration.}''
        \item Deployment Group: 

            A deployment group specify a set of instances to which a {\bf revision} of your application/code is deployed, according to deployment configurations.
        \item Revision:

            A {\bf snapshot} of your code-repository/application at a certain version.
    \end{itemize}
\end{frame}

\begin{frame}
    \frametitle{CodeDeploy - A quick tutorial}
    \framesubtitle{0. Steps}
    \begin{enumerate}
        \item Create IAM {\bf Roles} and {\bf Policies}
        \item Launch and configure your EC2 instances
        \item Create an {\bf Application} and {\bf Deployment Group}
        \item Prepare a {\bf Revision}
        \item Deploy!
    \end{enumerate}

    {\tiny You can also check out AWS's tutorial: \url{http://docs.aws.amazon.com/codedeploy/latest/userguide/getting-started-walkthrough.html}}
\end{frame}

\begin{frame}
    \frametitle{CodeDeploy - A quick tutorial}
    \framesubtitle{1. Configure IAM Roles and Policies}

    \begin{itemize}
        \item Our goal is to create {\bf two roles}, each attached with certain {\bf policies} (permitting some actions to be performed).
        \item Roles can be {\it assumed} by certain services ({\it e.g.} EC2, CodeDeploy). 
            By assuming the role, the service gains the attached {\bf policies}.
            \begin{itemize}
                \item {\it CodeDeploy} assumes some role to interact with the instances.
                \item Your {\it EC2 instances} assumes some other role to interact with {\it AWS S3}.
            \end{itemize}
        \pause
        \item Live Demo
    \end{itemize}
    {\tiny Reference: \url{http://docs.aws.amazon.com/codedeploy/latest/userguide/how-to-create-service-role.html}}
\end{frame}

\begin{frame}
    \frametitle{CodeDeploy - A quick tutorial}
    \framesubtitle{2. Launch EC2 Instances and Tag Them}
    \begin{itemize}
        \item Your code will be deployed to EC2 instances (on-demand, spot, reserved) launched in specific region(s).
        \item To let CodeDeploy discover your EC2 instances, {\bf tag them}.
            \begin{itemize}
                \item You can attach up to 10 tags to an instance.
                \item Tags are simply {\it key-value} pairs, {\it e.g.} {\tt deploy-r=true, app=shinydemo}.
                \item If you launch spot instances using the AWS Console, you {\bf must} tag instances after your requests are fulfilled.
            \end{itemize}
        \item Utilize the {\it user-data} of your EC2 instance to install CodeDeploy agent on the newly launched instances automatically.
    \end{itemize}
\end{frame}

\begin{frame}
    \frametitle{CodeDeploy - A quick tutorial}
    \framesubtitle{3. Create an Application and Deployment Group}
    \begin{itemize}
        \item Don't confuse the ``{\it application}'' here with your actual ``program'' or ``code-repository''. Here, ``application'' merely refers to a unique identifier in the CodeDeploy service.
        \item Deployment group uses {\bf tags} to identify a group of instances to which your program is deployed.
        \pause
        \item Live Demo
    \end{itemize}
    
    {\tiny Reference: \url{http://docs.aws.amazon.com/codedeploy/latest/userguide/how-to-create-application.html}}
\end{frame}

\begin{frame}[fragile]
    \frametitle{CodeDeploy - A quick tutorial}
    \framesubtitle{4. Prepare a Revision}
    \begin{itemize}
        \item A {\bf revision} refers to the collection of:
            \begin{enumerate}
                \item {\color{red} Your program (source code, executables, supporting libraries, configurations)}
                \item {\color{magenta} Deployment scripts/hooks}
                \item {\color{cyan} \tt appspec.yml}
            \end{enumerate}
        \item To prepare a revision, arrange files in a directory like this:
    \end{itemize}
    {\small
\begin{verbatim}
/home/alice/
\end{verbatim}
}
\vspace{-0.1in}
{\color{red}\small
\begin{verbatim}
  |--shinydemo
  |    |--server.R
  |    |--ui.R
  |    |--README.md
  |    |--static
  |    |     |--foo.js
\end{verbatim}
}
\vspace{-0.1in}
{\color{magenta}\small
\begin{verbatim}
  |--scripts 
  |    |--before_install.sh
  |    |--after_install.sh
\end{verbatim}
}
\vspace{-0.1in}
{\color{cyan}\small
\begin{verbatim}
  |--appspec.yml
\end{verbatim}
}


\end{frame}

\begin{frame}[fragile]
    \frametitle{CodeDeploy - A quick tutorial}
    \framesubtitle{4. Prepare a Revision (appspec.yml)}
    \vspace{-0.07in}
    \begin{columns}
        \begin{column}{0.4\textwidth}
            \begin{itemize}
                \item {\tt os}: {\tt linux} or {\tt windows}
                \item {\tt files}: a {\it mapping} to tell CodeDeploy agent, {\bf which files} to copy, and {\bf to where}
                \item {\tt permissions}: {\bf who} should own the files? what are the permissions?
                \item {\tt hooks}: what scripts should the CodeDeploy agent execute during the deployment
            \end{itemize}

        \end{column}
        \begin{column}{0.6\textwidth}
            {\footnotesize
            \begin{verbatim}
version: 0.0
os: linux

files:
 - source: shinydemo
   destination: /home/ubuntu

permissions:
  - object: shinydemo
    pattern: "**"
    owner: ubuntu
    mode: 644

hooks:
  BeforeInstall:
    - location: scripts/install_R.sh
      timeout: 180
  ApplicationStart:
    - location: scripts/run_shiny_server.sh
      timeout: 60
      runas: ubuntu
            \end{verbatim}
}
        \end{column}
    \end{columns}

    {\tiny YAML Syntax: \url{http://learn.getgrav.org/advanced/yaml}}\\
    {\tiny appspec.yml reference: \url{http://docs.aws.amazon.com/codedeploy/latest/userguide/app-spec-ref.html}}
\end{frame}

\begin{frame}
    \frametitle{CodeDeploy - A quick tutorial}
    \framesubtitle{4. Prepare a Revision (continued)}

    \begin{columns}
        \begin{column}{0.4\textwidth}
    \begin{itemize}
        \item {\bf Hooks} section defines scripts that will be executed during various lifecycle of a deployment process.
        \item You can hook up scripts with only the yellow events. 
        \item Using scripts can customize your deployment with a lot of flexibility.
    \end{itemize}
    \end{column}
    \begin{column}{0.6\textwidth}
    \includegraphics[width=\textwidth]{app_hooks.png}
    \end{column}
    \end{columns}

    \vspace{0.1in}
    \pause
    Live Demo

    \vspace{0.1in}
    {\tiny \url{http://docs.aws.amazon.com/codedeploy/latest/userguide/app-spec-ref.html\#app-spec-ref-hooks}}
\end{frame}

\begin{frame}
    \frametitle{CodeDeploy - A quick tutorial}
    \framesubtitle{5. Deploy and Monitor the Deploying Process}
    \begin{itemize}
        \item To deploy the prepared revision, we can put the directory into a {\tt .zip} or {\tt .tar.gz} archive.
        \item Upload the archive to AWS S3 or GitHub.
        \item Deploy it!
        \pause
        \item Live Demo
    \end{itemize}

    {\tiny To see how to deploy your code on GitHub, please refer to \url{http://docs.aws.amazon.com/codedeploy/latest/userguide/how-to-deploy-revision.html}}
\end{frame}

\begin{frame}
    \frametitle{Deploy the same program, but run with different parameters?}
    \begin{itemize}
        \item Use instance meta-data.
        \item Use instance tags and {\tt ec2-describe-tags}.
    \end{itemize}
\end{frame}

\end{document}
